\section*{Question 2}
    \subsection*{Constraints}
    \begin{itemize}
        \item \textbf{Low cost:} The cost of the system is an important constraint. The cost of the system should be as low as possible if large quantities will be produced.
        \item \textbf{Power consumption:} The power consumption of the system is an important constraint which has to be considered. If an embedded system is battery powered, the power consumption of the system should be as low as possible to make sure that the functionality of the system will be available for as long as possible. If the system is not battery powered, the power consumption of the system should be as low as possible to reduce the heat generated by the system.
        \item \textbf{Performance:} There might be performance constraints on the system. This might be to enhance user experience, make sure safety critical functions operate as soon as needed or to make sure that the system can handle the workload.
        \item \textbf{Size:} Size can be an important constraint since many embedded systems have to fit in tight spaces, such as electronic systems in cars, phones, planes, etc. Some embedded systems might also have to be portable, such as a hand held communication radio, headphones and calculators.
    \end{itemize}

    \subsection*{Functionality}
    The intended functionality of the embedded system has to be clearly defined. This is importand to know to decide on what scheduling algorithms to use, what hardware to use, what software to use, etc. The functionality of the system might also be constrained by the hardware and software used. For example, if the system is battery powered, the functionality of the system might be constrained by the power consumption of the system. If the system is a safety critical system, the functionality of the system might be constrained by the performance of the system.

    \subsection*{Real-time requirements}
    If there are real-time requirements, the system have to be designed to execute certain tasks in a deterministic manner and continually adapt to changes in the environment surrounding the system.