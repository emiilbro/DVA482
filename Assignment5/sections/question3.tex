\section{Question 3}

    To calculate the maximum execution time for Task A to make the task set schedulable, the first thing to try is to make the statement $U <= n(2^{\frac{1}{n}} - 1)$ true. In this case $n = 3$ which gives us $U <= 0.78$. So continuing from there we can find the maximum $C_A$ using the equation $U = \sum_{i=1}^{n} \frac{C_i}{T_i}$. The following table contains the given values for period and execution time for the tasks.
    \renewcommand{\arraystretch}{1.4}
        \begin{figure}[H]
        \centering
        \begin{minipage}{0.5\textwidth}
            \begin{table}[H]
            \centering
            \begin{tabular}{|l|l|l|}
                \hline
                \textbf{Task}   & \textbf{T=D}  & \textbf{C}  \\ \hline
                A               & 4             & $C_A$       \\ \hline
                B               & 12            & 4           \\ \hline
                C               & 20            & 9           \\ \hline
            \end{tabular}
            \end{table}
        \end{minipage}%
        \caption{Task set}
        \label{fig:Q3tasks}
        \end{figure}
    \renewcommand{\arraystretch}{1.0}

    Using the values in the table and the formula $U = \sum_{i=1}^{n} \frac{C_i}{T_i}$ we can extract $C_A$ and get the following equation: $C_A = 4*(0.78 - \frac{4}{12} - \frac{9}{20}) = -0.014ms$. This is not a valid value for $C_A$, so we need to try another approach.\\
    
    We will now try a different approach by finding the worst case execution time (WCET) for task C which is the lowest priority task according to RM scheduling protocols. WCET for task C cannot be slower than $20ms$ because then it will miss its deadline. WCET is found by releasing all higher priority tasks at the same time as task C, but since we do not know the execution time of task A we will only release task B and C and then find the remaining execution time within the $20ms$ period. Within the $20ms$ time period task B will be released twice and execute a total of $2*4ms = 8ms$, task C will be released once and execute a total of $9ms$. If we add up the execution time of task B and C we get $8ms + 9ms = 17ms$. This means that the remaining execution time for task A is a total of $20ms - 17ms = 3ms$. To find the maximum execution time for task A we will have to divide the remaining free time in the $20ms$ period with the number of times task A is released. Task A is released every $4ms$ which means that it will be released $5$ times within the $20ms$ period. The maximum execution time for task A is then $\frac{3ms}{5} = 0.6ms$. So the maximum execution time for task A is $0.6ms$ to make the task set schedulable.\\

    % We will now try to find maximum $C_A$ by calculating the total amount of execution time needed for all tasks in one hyperperiod. The hyperperiod, or the lowest common multiple (LCM) of the periods of the tasks, is $60ms$. We need to find how many instances occur of each task and then multiply it by the execution time of each task.\\
    % Instances of each task in one hyperperiod:
    % \begin{itemize}
    %     \item Instances of task A: $\frac{LCM}{Ta} = \frac{60}{4} = 15$
    %     \item Instances of task B: $\frac{LCM}{Tb} = \frac{60}{12} = 5$
    %     \item Instances of task C: $\frac{LCM}{Tc} = \frac{60}{20} = 3$
    % \end{itemize}
    
    % The total amount of execution time needed for all tasks in one hyperperiod is then:
    % \begin{itemize}
    %     \item Task A: $15*Ca =?$
    %     \item Task B: $4*5 = 20ms$
    %     \item Task C: $9*3 = 27ms$
    % \end{itemize}

    % We now have the following equation: $15*Ca + 20 + 27 <= 60$. Solving for $C_A$ gives us $Ca <= \frac{60-27-20}{15} = 0.87ms$. This gives us a maximum execution time for task A of $0.87ms$ to make the task set schedulable.\\
